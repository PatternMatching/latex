\documentclass[11pt]{article}
\usepackage{amsmath}
\usepackage{amsthm}
\usepackage{amssymb}
\usepackage{enumerate}
\usepackage{MnSymbol}
\usepackage{algorithm,algpseudocode}
\usepackage{parskip}
\usepackage{tikz}
\usetikzlibrary{fit,positioning}
\setlength{\oddsidemargin}{0in}
\setlength{\evensidemargin}{0in}
\setlength{\textheight}{9in}
\setlength{\textwidth}{6.5in}
\setlength{\topmargin}{-0.5in}
\setlength\parindent{0pt}

% Sample macros -- how you define new commands
% My own set of frequently-used macros have grown to many hundreds of lines.
% Here are some simple samples.

\newcommand{\Adv}{{\mathbf{Adv}}}       
\newcommand{\prp}{{\mathrm{prp}}}                  % How to define new commands 
\newcommand{\calK}{{\cal K}}
\newcommand{\outputs}{{\Rightarrow}}                
\newcommand{\getsr}{{\:\stackrel{{\scriptscriptstyle\hspace{0.2em}\$}}{\leftarrow}\:}}
\newcommand{\andthen}{{\::\;\;}}    % \, \: \; for thinspace, medspace, thickspace
\newcommand{\Rand}[1]{{\mathrm{Rand}[{#1}]}}       % A command with one argument
\newcommand{\Perm}[1]{{\mathrm{Perm}[{#1}]}}       
\newcommand{\Randd}[2]{{\mathrm{Rand}[{#1},{#2}]}} % and with two arguments
\newcommand{\E}[1]{{\mathrm{E}\left[{#1}\right]}}
\newcommand{\Cov}[2]{{\mathrm{Cov}[{#1},{#2}]}}
\newcommand{\Var}[1]{{\mathrm{Var}[{#1}]}}
\newcommand{\tab}[1]{\hspace{.2\textwidth}\rlap{#1}}
\newcommand{\itab}[1]{\hspace{0em}\rlap{#1}}

\newcommand\independent{\protect\mathpalette{\protect\independenT}{\perp}}
\def\independenT#1#2{\mathrel{\rlap{$#1#2$}\mkern2mu{#1#2}}}

%%%%%%%%%%%%%%%%%%%%%%%%%%%%%%%%%%%%%%%%%%%%%%%%%%%%%%%%%%%%%%%%%%%%%%%%%%%
\title{\bf Midterm Exam Submission\\[2ex] 
       \rm\normalsize EN605.725 --- Fall 2014}
\date{\today}
\author{\bf Peter Hennings}

\begin{document}
\maketitle

%%%%%%%%%%%%%%%%%%%%%%%%%%%%%%%%%%%%%%%%%%%%%%%%%%%%%%%%
\section*{Problem 1} 

\begin{enumerate}[(a)]

\item We'd like to show the following:

\begin{align*}
& {n \choose 2} = {k \choose 2} + k(n-k) + {n-k \choose 2}
\end{align*}

\begin{proof}

If we rewrite the above using factorials, you have:

\begin{align*}
& \frac{n!}{2(n-2)!} = \frac{k!}{2(k-2)!} + k(n-k) + \frac{(n-k)!}{2(n-k-2)!}
\end{align*}

Expanding the above, we can simplify the right hand side a bit to enable combining:

\begin{align*}
\frac{n!}{2(n-2)!} &= \frac{k(k-1)(k-2)!}{2(k-2)!} + \frac{2k(n-k)}{2} + \frac{(n-k)(n-k-1)(n-k-2)!}{2(n-k-2)!} \\
&= \frac{k(k-1)}{2} + \frac{2k(n-k)}{2} + \frac{(n-k)(n-k-1)}{2} \\
&= \frac{(2k+1+n-k-1)(n-k) + k(k-1)}{2} \\
&= \frac{(n+k-1)(n-k) + k(k-1)}{2} \\
&= \frac{n(n-1)}{2}
\end{align*}

Finally, multiplying by a convenient form of the identity, we arrive at:

\begin{align*}
\frac{n!}{2(n-2)!} &= \frac{n(n-1)}{2} \frac{(n-2)!}{(n-2)!} \\
&= \frac{n!}{2!(n-2)!}
\end{align*}

\end{proof}

\item We would like to verify, via mathematical induction, the following:

\begin{align*}
  & \sum_{k=1}^n (-1)^{k+1} k {n \choose k} = 0
\end{align*}

\begin{proof}
Basis: Let $n=2$.  Then we have:

\begin{align*}
  & \sum_{k=1}^2 (-1)^{k+1} k {2 \choose k} = (-1)^2 (1) \frac{2!}{1!(2-1)!} + (-1)^3 (2) \frac{2!}{2!(2-2)!} = 0
\end{align*}

So we have verified that the relation is true for $n=2$.

Induction Step: Assume that the basis is true for $n=i$.  Now, let $n=i+1$.  We have:

\begin{align*}
  & \sum_{k=1}^{i+1} (-1)^{k+1} k {i+1 \choose k} = \sum_{k=1}^{i+1} (-1)^{k+1} k \frac{(i+1)!}{k!(i+1-k)!}
\end{align*}

\end{proof}

\end{enumerate}

%%%%%%%%%%%%%%%%%%%%%%%%%%%%%%%%%%%%%%%%%%%%%%%%%%%%%%%%
\section*{Problem 2}


%%%%%%%%%%%%%%%%%%%%%%%%%%%%%%%%%%%%%%%%%%%%%%%%%%%%%%%%
\section*{Problem 3}


%%%%%%%%%%%%%%%%%%%%%%%%%%%%%%%%%%%%%%%%%%%%%%%%%%%%%%%%
\section*{Problem 4}


%%%%%%%%%%%%%%%%%%%%%%%%%%%%%%%%%%%%%%%%%%%%%%%%%%%%%%%%
\section*{Problem 5}


%%%%%%%%%%%%%%%%%%%%%%%%%%%%%%%%%%%%%%%%%%%%%%%%%%%%%%%%
\section*{Problem 6}


%%%%%%%%%%%%%%%%%%%%%%%%%%%%%%%%%%%%%%%%%%%%%%%%%%%%%%%%
\section*{Problem 7}


%%%%%%%%%%%%%%%%%%%%%%%%%%%%%%%%%%%%%%%%%%%%%%%%%%%%%%%%
\section*{Problem 8}


%%%%%%%%%%%%%%%%%%%%%%%%%%%%%%%%%%%%%%%%%%%%%%%%%%%%%%%%
\section*{Problem 9}



\end{document}
