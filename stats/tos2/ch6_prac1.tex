\documentclass[11pt]{article}
\usepackage{amsmath}
\usepackage{amsthm}
\usepackage{amssymb}
\usepackage{enumerate}
\usepackage{MnSymbol}
\usepackage{algorithm,algpseudocode}
\usepackage{parskip}
\usepackage{tikz}
\usetikzlibrary{fit,positioning}
\setlength{\oddsidemargin}{0in}
\setlength{\evensidemargin}{0in}
\setlength{\textheight}{9in}
\setlength{\textwidth}{6.5in}
\setlength{\topmargin}{-0.5in}
\setlength\parindent{0pt}

% Sample macros -- how you define new commands
% My own set of frequently-used macros have grown to many hundreds of lines.
% Here are some simple samples.

\newcommand{\Adv}{{\mathbf{Adv}}}       
\newcommand{\prp}{{\mathrm{prp}}}                  % How to define new commands 
\newcommand{\calK}{{\cal K}}
\newcommand{\outputs}{{\Rightarrow}}                
\newcommand{\getsr}{{\:\stackrel{{\scriptscriptstyle\hspace{0.2em}\$}}{\leftarrow}\:}}
\newcommand{\andthen}{{\::\;\;}}    % \, \: \; for thinspace, medspace, thickspace
\newcommand{\Rand}[1]{{\mathrm{Rand}[{#1}]}}       % A command with one argument
\newcommand{\Perm}[1]{{\mathrm{Perm}[{#1}]}}       
\newcommand{\Randd}[2]{{\mathrm{Rand}[{#1},{#2}]}} % and with two arguments
\newcommand{\E}[1]{{\mathrm{E}\left[{#1}\right]}}
\newcommand{\Cov}[2]{{\mathrm{Cov}[{#1},{#2}]}}
\newcommand{\Var}[1]{{\mathrm{Var}[{#1}]}}
\newcommand{\tab}[1]{\hspace{.2\textwidth}\rlap{#1}}
\newcommand{\itab}[1]{\hspace{0em}\rlap{#1}}

\newcommand\independent{\protect\mathpalette{\protect\independenT}{\perp}}
\def\independenT#1#2{\mathrel{\rlap{$#1#2$}\mkern2mu{#1#2}}}

%%%%%%%%%%%%%%%%%%%%%%%%%%%%%%%%%%%%%%%%%%%%%%%%%%%%%%%%%%%%%%%%%%%%%%%%%%%
\title{\bf Practice Set 1 Solutions\\[2ex] 
       \rm\normalsize EN605.726 --- Spring 2015}
\date{\today}
\author{\bf Peter Hennings}

\begin{document}
\maketitle

%%%%%%%%%%%%%%%%%%%%%%%%%%%%%%%%%%%%%%%%%%%%%%%%%%%%%%%%
\section*{Problem 1.1} 

Let $X_i (i = 1,2,\dots,n)$ be a random sample from a distribution with the pdf:

\begin{align*}
  f(x \mid \theta) &= \theta x^{\theta-1}, \theta > 0, 0 < x < 1
\end{align*}

We would like to verify that the distribution of $X$ is a member of the single parameter exponential family.  Also, we will find the pdf of $Y = -\ln{X}$, utilizing the transformation method, and identify the distribution of $\sum_i Y_i$ through the moment generating function of $Y$.

To show that the above pdf is that of a single-parameter exponential, it must have the following form:

\begin{align*}
  f(x \mid \theta) &= h(x)c(\theta) \exp{\left(  w(\theta) t(x) \right)}
\end{align*}

We can rewrite the above pdf as follows:
\begin{align*}
  f(x \mid \theta) &= \theta \exp{(\ln{(x)}(\theta-1))}
\end{align*}

Therefore, we have verified that $X$ is, in fact, a member of the single parameter exponential family.  Utilizing the transformation method, we proceed to calculate the pdf of $Y$ as specified.  Specifically, we have that $g(x) = -\ln{x}$ and $g^{-1}(y) = e^{-y}$.  From Theorem 2.1.5, we have that insofar as $g(x)$ is a monotone function and $f_X$ is continuous:

\begin{align*}
  f_Y(y) &= 
           \begin{cases}
             f_X(g^{-1}(y)) \left| \frac{d}{dy} g^{-1}(y) \right| & y \in \mathcal{Y} \\
             0 & \text{otherwise}
           \end{cases} \\
  &= 
    \begin{cases}
      \theta (e^{-y})^{(\theta-1)} \left| -e^{-y} \right| & y \in \mathcal{Y} \\
      0 & \text{otherwise}
    \end{cases} \\
  &= 
    \begin{cases}
      \theta e^{-\theta y} & y \in \mathcal{Y} \\
      0 & \text{otherwise}
    \end{cases}
\end{align*}

We see that the resultant pdf is of an exponential distribution with parameter $\theta$.  The moment generating function of $Y$ is then $M(t) = \frac{1}{1-t/\theta}$.  If we let $Z = \sum_i Y_i$ and assume the $Y$'s are independent, then we have:

\begin{align*}
  M_Z(t) &= \prod_i M_{Y_i} = \left( \frac{1}{1-t/\theta} \right)^n
\end{align*}

We observe that this is the moment generating function of a $\Gamma(n, \theta)$ distribution.



\end{document}
