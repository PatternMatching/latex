\documentclass[11pt,a4paper]{report}
\usepackage{amsmath,amsfonts,amssymb,amsthm,epsfig,epstopdf,titling,url,array}
\usepackage{enumerate} \usepackage{parskip}

\makeatletter \def\thm@space@setup{% \thm@preskip=\parskip
\thm@postskip=0pt } \makeatother

\title{\bf Notes\\[2ex] \rm\normalsize EN605.725 --- Fall 2014}
\date{\today} \author{\bf Peter Hennings}

\theoremstyle{definition} \newtheorem{defn}{Definition}[section]

\theoremstyle{theorem} \newtheorem{thm}{Theorem}[section]

\theoremstyle{example} \newtheorem{ex}{Example}[section]

\begin{document}
\maketitle

\chapter{Probability Theory}

\section{Set Theory}

Statistics builds on the foundation of probability theory.  Similarly,
probability theory in turn builds upon set theory.

A main objective of a statistician is to draw conclusions about a
population of objects having conducted an experiment.  The first step
in this process is identifying the possible outcomes of said
experiment.

\begin{defn} The set, $S$, of all possible outcomes of a particular
experiment is called the \textit{sample space} for the experiment.
\end{defn}

We can classify sample spaces into two types: countable or
uncountable.  If the elements of a sample space can be put into a 1-1
correspondence with a subset of the integers, then the sample space is
countable.  Otherwise, it is uncountable.

\begin{defn} An \textit{event} is any collection of possible outcomes
of an experiment, that is, any subset of $S$ (including $S$ itself).
\end{defn}

Let $A$ be an event, a subset of $S$.  We say that the event $A$
occurs if the outcome of the experiment is the set $A$.  When speaking
of probabilities, we generally speak of the probability of an event,
rather than a set.  But we may use the terms interchangeably.

We define formally the following two relationships, which allow us to
order and equate sets:

\begin{align*} & A \subset B \Leftrightarrow x \in A \Rightarrow x \in
B & \textrm{(containment)} \\ & A = B \Leftrightarrow A \subset B
\textrm{ and } B \subset A & \textrm{(equality)}
\end{align*}

Given any two events (sets) $A$ and $B$, we have the following
elementary set operations:

\textbf{Union}: The union of $A$ and $B$, written $A \cup B$, is the
set of elements that belong to either $A$ or $B$ or both:

\begin{align*} & A \cup B = \{x:x \in A \textrm{ or } x \in B\}
\end{align*}

\textbf{Intersection}: The intersection of $A$ and $B$, written $A
\cap B$, is the set of elements that belong to both $A$ and $B$:

\begin{align*} & A \cap B = \{x:x \in A \textrm{ and } x \in B\}
\end{align*}

\textbf{Complementation}: The complement of $A$, written
$A^{\complement}$, is the set of all elements that are not in A:

\begin{align*} & A^{\complement} = \{x:x \notin A\}
\end{align*}

\begin{thm} For any three events, A, B, and C, defined on a sample
space S,
\begin{align*} & A \cup B = B \cup A & \textrm{(commutativity)} \\ & A
\cap B = B \cap A \\ & A \cup (B \cup C) = (A \cup B) \cup C &
\textrm{(associativity)} \\ & A \cap (B \cap C) = (A \cap B) \cap C \\
& A \cap (B \cup C) = (A \cap B) \cup (A \cap C) &
\textrm{(distributive laws)} \\ & A \cup (B \cap C) = (A \cup B) \cap
(A \cup C) \\ & (A \cup B)^{\complement} = A^{\complement} \cap
B^{\complement} & \textrm{(DeMorgan's Laws)} \\ & (A \cap
B)^{\complement} = A^{\complement} \cup B^{\complement}
\end{align*}
\end{thm}

\begin{defn} Two events $A$ and $B$ are disjoint (or mutually
exclusive) if $A \cap B = \emptyset$.  The events $A_1, A_2, \dots$
are pairwise disjoint (or mutually exclusive) if $A_i \cap A_j =
\emptyset$ for all $i \not = j$.
\end{defn}

Disjoint sets are sets with no points in common.  If we draw a Venn
diagram for two disjoint sets, the sets do not overlap.

\begin{defn} If $A_1, A_2, \dots$ are pairwise disjoint and
$\cup_{i=0}^{\infty} A_i = S$, then the collection $A_1, A_2, \dots$
forms a partition of $S$.
\end{defn}

Partitions are very useful, allowing us to divide the sample space
into small, nonoverlapping pieces.

\section{Basics of Probability Theory}

The realization of an experiment is an outcome in the sample space.
If the experiment is performed a number of times, different outcomes
may occur each time or some outcomes may repeat.  Frequency of
occurrence is effectively probability.  More probable outcomes occur
more frequently.

\begin{defn} A collection of subsets of $S$ is called a sigma algebra
(or Borel field), denoted by $\mathcal{B}$, if it satisfies the
following three properties:

\begin{enumerate}[(a)]
  \item $\emptyset \in \mathcal{B}$ (the empty set is an element of
$\mathcal{B}$)
  \item If $A \in \mathcal{B}$ then $A^{\complement} \in \mathcal{B}$
($\mathcal{B}$ is closed under complementation)
  \item If $A_1, A_2, \dots \in \mathcal{B}$, then
$\cup_{i=1}^{\infty} A_i \in \mathcal{B}$
\end{enumerate}

\end{defn}

The empty set $\emptyset$ is a subset of any set.  Thus, $\emptyset
\subset S$.  A sigma algebra always contains this subset.  Since $S =
\emptyset^{\complement}$, properties (a) and (b) imply that $S$ is
always in $\mathcal{B}$.  $\mathcal{B}$ is closed under countable
intersection from DeMorgan's laws.

Many different sigma algebras can be associated with a sample space
$S$.  For example, the collection of the two sets $\{\emptyset, S\}$
is usually called the trivial sigma algebra.  The only sigma algebra
we will be concerned with is the smallest one that contains all of the
open sets in a given sample space $S$.

If $S$ is finite or countable, then these technicalities do not arise,
for we define for a given sample space $S$,

\begin{align*} & \mathcal{B} = \{ \textrm{all subsets of S, including
S itself} \}
\end{align*}

If $S$ has $n$ elements, there are $2^n$ sets in $\mathcal{B}$.  In
general, if $S$ is uncountable, it is not an easy task to describe
$\mathcal{B}$.  However, $\mathcal{B}$ is chosen to contain any set of
interest.

\begin{defn} Given a sample space $S$ and an associated sigma algebra
$\mathcal{B}$, a probability function is a function $P$ with domain
$\mathcal{B}$ that satisfies:

\begin{enumerate}
 \item $P(A) \ge 0$ for all $A \in B$
 
 \item $P(S) = 1$

 \item If $A_1, A_2, \dots \in \mathcal{B}$ are pairwise disjoint,
then $P\left( \cup_{i=1}^{\infty} A_i \right) = \sum_{i=1}^{\infty}
P(A_i)$.
\end{enumerate}

\end{defn}

The above properties are generally referred to as the axioms of
probability (or the Kolmogorov Axioms).  Any function $P$ that
satisfies the Axioms of Probability is called a probability function.
The axiomatic definition makes no attempt to tell what particular
function $P$ to choose; it merely requires $P$ to satisfy the axioms.
For any sample space many different probability functions can be
defined.

Consider the simple experiment of tossing a fair coin, so $S = \{H,
T\}$.  By a ``fair'' coin we mean a balanced coin that is equally
likely to land on heads as it is tails ($P(\{H\}) = P(\{T\})$).  We
note the following:

\begin{align*} & P(\{H\}) = P(\{T\}) \\ & P(\{H\}) + P(\{T\}) = 1 \\ &
P(\{H\}) = P(\{T\}) = 0.5
\end{align*}

\begin{thm} Let $S = \{s_1, \dots, s_n\}$ be a finite set.  Let
$\mathcal{B}$ be any sigma algebra of subsets of $S$.  Let $p_1,
\dots, p_n$ be nonnegative numbers that sum to 1.  For any $A \in
\mathcal{B}$, define $P(A)$ by:

  \begin{align*} & P(A) = \sum_{\{i:s_i \in A\}} p_i
  \end{align*}

\end{thm}

\begin{proof} We will give the proof for finite $S$.  For any $A \in
\mathcal{B}$, $P(A) = \sum_{\{i:s_i \in A\}} p_i \ge 0$, because every
$p_i \ge 0$.  Thus, Axiom 1 is true.  Now,

\begin{align*} & P(S) = \sum_{\{i:s_i \in A\}} p_i = \sum_{i=1}^n p_i
= 1
\end{align*}

Thus, Axiom 2 is true.  Let $A_1, \dots, A_k$ denote pairwise disjoint
events.  ($\mathcal{B}$ contains only a finite number of sets, so we
need consider only finite disjoint unions).  Then,

\begin{align*} & P \left( \bigcup_{i=1}^k A_i \right) = \sum_{\{j:s_j
\in \cup_{i=1}^k A_i \}} p_j = \sum_{i=1}^k \sum_{\{j:s_j \in
\cup_{i=1}^k A_i \}} p_j = \sum_{i=1}^k P(A_i)
\end{align*}

The first and third equalities are true by the definition of $P(A)$.
The disjointedness of the $A_i$'s ensures that the second equality is
true, because the same $p_j$'s appear exactly once on each side of the
equality.  Thus Axiom 3 is true and Kolmogorov's Axioms are satisfied.

\end{proof}

\subsection{The Calculus of Probabilities}

\begin{thm} If $P$ is a probability function and $A$ is any set in
$\mathcal{B}$, then

\begin{enumerate}[(a)]

\item $P(\emptyset) = 0$, where $\emptyset$ is the empty set

\item $P(A) \le 1$

\item $P(A^{\complement}) = 1 - P(A)$

\end{enumerate}

\end{thm}

\chapter{Transformations and Expectations}

\chapter{Common Families of Distributions}

\section{Discrete Distributions}

A random variable $X$ is said to have a discrete distribution if the
range of $X$, the sample space, is countable.  In most of these
situations, the random variable has integer-valued outcomes.

\textit{Discrete Uniform Distribution}

A random variable $X$ has a discrete uniform $(1,N)$ distribution if

\begin{align*} P(X = x \, | \, N) &= \frac{1}{N}, & x = 1, 2, \dots, N
\end{align*}

where $N$ is a specified integer.  This distribution puts equal mass
on each of the outcomes $1,2,\dots,N$.

When we deal with parametric distributions, as is mostly the case, the
distribution is dependent on values of the parameters.  In order to
emphasize this fact and to keep track of params, we write them in the
pmf preceded by a ``$|$'' (given).

To calculate the mean and variance of $X$, recall the identities
(provable by induction)

\begin{align*} \sum_{i=1}^k &= \frac{k(k+1)}{2} \text{ and } \\
\sum_{i=1}^k i^2 &= \frac{k(k+1)(2k+1)}{6}
\end{align*}

We then have

\begin{align*} \mathrm{E} \left[ X \right] &= \sum_{i=1}^N x P(X = x
\, | \, N) = \sum_{i=1}^N x \frac{1}{N} = \frac{N+1}{2} \\ \mathrm{E}
\left[ X^2 \right] &= \sum_{i=1}^N x^2 \frac{1}{N} =
\frac{(N+1)(2N+1)}{6} \\ \mathrm{Var}(X) &= \mathrm{E} \left[ X^2
\right] - \left( \mathrm{E} \left[ X \right] \right)^2 \\ &=
\frac{(N+1)(2N+1)}{6} - \left( \frac{N+1}{2} \right)^2 \\ &=
\frac{(N+1)(N-1)}{12}
\end{align*}

\textit{Hypergeometric Distribution}

The hypergeometric distribution has many applications in finite
population sampling and is best understood through the classic example
of the urn model.

Suppose we have a large urn filled with $N$ balls that are identical
in every way except that $M$ are red and $N-M$ are green.  We reach
in, blindfolded, and select $K$ balls at random (the $K$ balls are
taken all at once, a case of sampling without replacement).  What is
the probability that exactly $x$ are red?

The total number of samples of size $K$ that can be drawn from the $N$
balls is ${N \choose K}$.  It is required that $x$ of the balls be
red, and this can be accomplished in ${M \choose x}$ ways, leaving
${N-M \choose K-x}$ ways of filling out the sample with $K-x$ green
balls.  Thus, if we let $X$ denote the number of red balls in a sample
of size $K$, then $X$ has a hypergeometric distribution given by:

\begin{align*} P(X = x \, | \, N, M, K) &= \frac{{M \choose K} {N-M
\choose K-x}}{{N \choose K}}
\end{align*}

The mean is given by

\begin{align*} \mathrm{E} \left[ X \right] &= \sum_{i=0}^K x \frac{{M
\choose x} {N-M \choose K-x}}{{N \choose K}} = \sum_{i=1}^K x \frac{{M
\choose x} {N-M \choose K-x}}{{N \choose K}}
\end{align*}

Rewriting as a summation from zero and using the combinatoric
identities from Chapter 2, we simplifyto obtain:

\begin{align*} \mathrm{E} \left[ X \right] &= \sum_{x=1}^K \frac{M
{M-1 \choose x-1}{N-m \choose K-x}}{\frac{N}{K}{N-1 \choose K-1}} =
\frac{KM}{N} \sum_{i=1}^K \frac{{M-1 \choose x-1}{N-M \choose
K-x}}{{N-1 \choose K-1}} \\ &= \frac{KM}{N}
\end{align*}

Similarly, for the variance:

\begin{align*} \mathrm{Var} X &= \frac{KM}{N} \left(
\frac{(N-M)(N-K)}{N(N-1)} \right)
\end{align*}

\begin{ex}
  The hypergeometric distribution has application in acceptance sampling, as this example will illustrate.  Suppose a retailer buys goods in lots and each item can be either acceptable or defective.  Let
  \begin{center}
    $N = $ \# of items in a lot, \\
    $M = $ \# of defectives in a lot
  \end{center}

Then we can calculate the probability that a sample of size $K$
contains $x$ defectives.  To be specific, suppose that a lot of 25
machine parts is delivered, where a part is considered acceptable only
if it passes tolerance.  We sample 10 parts and find that none are
defective (all are within tolerance).  What is the probability of this
event if there are 6 defectives in the lor of 25?  Applying the
hypergeometric distribution with $N=25, M=6, K=10$, we have

\begin{align*} P(X=0) &= \frac{{6 \choose 0}{19 \choose 10}}{{25
\choose 10}} = 0.028
\end{align*}

showing that our observed event is quite unlikely if there are 6 (or
more!) defectives in the lot.
\end{ex}

\textit{Binomial Distribution}

The binomial distribution, one of the more useful discrete
distributions, is based on the idea of a Bernoulli trial.  A Bernoulli
trial (named for James Bernoulli, one of the founding fathers of
probability theory) is an experiment with two, and only two, possible
outcomes.  A random variable $X$ has a \textit{Bernoulli}$(p)$
distribution if

\begin{align*} X &=
  \begin{cases} 1 & \text{with probability } p \\ 0 & \text{with
probability } 1-p
  \end{cases}
\end{align*}

Mean and variance:

\begin{align*}
  \mathrm{E} \left[ X \right] &= 1p + 0(1-p) = p \\
  \mathrm{Var} X &= (1-p)^2 p + (0-p)^2(1-p) = p(1-p)
\end{align*}

If $n$ Bernoulli trials are performed, define the events

\begin{align*}
  A_i &= \{ X=1 \text{ on the ith trial }\}, i=1,2,\dots,n
\end{align*}

We assume the events are a collection of independent events (as is the case in coin tossing).  We can then derive the distribution of the total number of successes in $n$ trials.  Define a random variable $Y$ by

\begin{align*}
  Y = \text{ total number of successes in n trials}
\end{align*}

The event $\{Y = y\}$ will occur only if, out of the events $A_1, \dots, A_n$, exactly $y$ of them occur, and necessarily $n-y$ of them do not occur.  One particular outcome (one particular ordering of occurrences and nonoccurences) of the $n$ Bernoulli trials might be $A_1 \cap A_2 \cap A_3^{\complement} \cap \dots \cap A_{n-1} \cap A_n^{\complement}$.  This has probability of occurrence

\begin{align*}
  \Pr{A_1 \cap A_2 \cap A_3^{\complement} \cap \dots \cap A_{n-1} \cap A_n^{\complement}} &= pp(1-p) \dots p(1-p) \\
  &= p^y(1-p)^{n-y}
\end{align*}

where we have used the independence of the $A_i$'s in this calculation.  Notice that the calculation is not dependent on which set of $y$ $A_i$'s occurs, only that some set of $y$ occurs.  Furthermore, the event $\{Y=y\}$ will occur no matter which set of $y$ $A_i$'s occurs.  Putting this all together, we see that a particular sequence of $n$ trials with exactly $y$ successes has probability $p^y(1-p)^{n-y}$ of occurring.  Since there are ${n \choose y}$ such sequences (the number of orderings of $y$ 1's and $n-y$ 0's), we have:

\begin{align*}
  \Pr{Y=y \, | \, n, p} &= {n \choose y} p^y(1-p)^{n-y}
\end{align*}

and $Y$ is called a \textit{binomial}(n,p) random variable.

\begin{thm}
  (Binomial Theorem) For any real numbers $x$ and $y$ and integer $n \ge 0$,
  \begin{align*}
    (x+y)^n &= \sum_{i=0}^n {n \choose i} x^i y^{n-i}
  \end{align*}
\end{thm}

\begin{ex} (Dice probabilities)
  Suppose we are interested in finding the probability of obtaining at least one 6 in four rolls of a fair die.  This experiment can be modeled as a sequence of four Bernoulli trials with success probability $p = 1/6$.  Define the random variable $X$ by

  \begin{center}
    $X = $ total number of 6's in four rolls
  \end{center}

  Then $X \sim $ binomial$(4,\frac{1}{6})$ and

  \begin{align*}
    \Pr{\text{at least one 6}} &= \Pr{X > 0} = 1 - \Pr{X = 0} \\
    &= 1 - {4 \choose 0} \left( \frac{1}{6} \right)^0 \left( \frac{5}{6} \right) \\
    &= 1 - \left( \frac{5}{6} \right)^4 \\
    &= 0.518
  \end{align*}

  Consider another game; throw a pair of dice 24 times and ask for the probability of at least one double 6.

  \begin{align*}
    p &= \Pr{\text{roll a double 6}} = \frac{1}{36}
  \end{align*}

  So, if $Y =$ number of double 6's in 24 rolls, $Y \sim$ binomial$(24, \frac{1}{36})$ and

  \begin{align*}
    \Pr{\text{at least one double 6}} &= \Pr{Y > 0} \\
    &= 1 - \Pr{Y = 0} \\
    &= 1 - {24 \choose 0} \left( \frac{1}{36} \right)^0 \left( \frac{35}{36} \right)^{24} \\
    &= 1 - \left( \frac{35}{36} \right)^24 \\
    &= 0.491
  \end{align*}
\end{ex}

\textit{Poisson Distribution}

The Poisson distribution is a widely applied discrete distribution and can serve as a model for a number of different types of experiments.  For example, if we are modeling a phenomenon in which we are waiting for an occurrence (such as waiting for a bus, waiting for customers to arrive at a bank), the number of occurrences in a given time interval can sometimes be modeled by the Poisson distribution.  One of the basic assumptions on which the Poisson distribution is built is that, for small time intervals, the probability of an arrival is proportional to the length of the waiting time.  In other words, the longer we wait, the more likely it is that a customer will enter a bank.

Another area of application is in spatial distributions, where, for example, the Poisson may be used to model the distribution of bomb hits in an area or the distribution of fish in a lake.

The Poisson has a single parameter $\lambda$, sometimes called the intensity parameter.  A random variable $X$, taking values in the nonnegative integers, has a Poisson$(\lambda)$ distribution if

\begin{align*}
  \Pr{(X = x \, | \, \lambda)} &= \frac{e^{-\lambda} \lambda^x}{x!}, x = 0, 1, \dots
\end{align*}

To see that $\sum_{x=0}^{\infty} \Pr{(X = x \, | \, \lambda)} = 1$, recall the Taylor series expansion of $e^y$,

\begin{align*}
  e^y &= \sum_{i=0}^{\infty} \frac{y^i}{i!}
\end{align*}

Thus,

\begin{align*}
  \sum_{x=0}^{\infty} \Pr{(X = x \, | \, \lambda)} &= e^{\lambda} \sum_{x=0}^{\infty} \frac{\lambda^x}{x!} = e^{-\lambda} e^{\lambda} = 1
\end{align*}

The mean and variance of a Poisson can both be shown to be equal to the parameter, $\lambda$:

\begin{align*}
  \mathrm{E} \left[ X \right] &= \sum_{x=0}^{\infty} x \frac{e^{-\lambda} \lambda^x}{x!} \\
  &= \sum_{x=1}^{\infty} x \frac{e^{-\lambda} \lambda^x}{x!} \\
  &= \lambda e^{-\lambda} \sum_{x=1}^{\infty} x \frac{\lambda^{x-1}}{(x-1)!} \\
  &= \lambda e^{-\lambda} \sum_{y=0}^{\infty} y \frac{\lambda^y}{y!} \\
  &= \lambda e^{-\lambda} e^{\lambda} = \lambda
\end{align*}

\chapter{Multiple Random Variables}

\section{Joint and Marginal Distributions}

All models covered so far have been univariate and involve only one
random variable.  Models in this chapter will be multivariate and
involve more than one random variable.

\begin{defn} An n-dimensional random vector is a function from a
sample space $S$ into $\mathcal{R}^n$, n-dimensional Euclidean space.
\end{defn}

\begin{defn} Let $(X,Y)$ be a discrete bivariate random vector.  Then
the function $f(x,y)$ from $\mathcal{R}^2$ into $\mathcal{R}$ defined
by $f(x,y) = P(X = x, Y = y)$ is called the joint probability mass
function or joint pmf of $(X,Y)$.
\end{defn}

Continuing with our bivariate random vector $(X,Y)$, if we let $A$ be
any subset of $\mathcal{R}^2$ then:

\begin{align*} P((X,Y) \in A) &= \sum_{(x,y) \in A} f(x,y)
\end{align*}

Since $(X,Y)$ is discrete, $f(x,y)$ is nonzero for at most a countable
number of points.  Thus, the sum is a countable sum even if $A$ is an
uncountable set.

\begin{defn} A function $f(x,y)$ from $\mathcal{R}^2$ into
$\mathcal{R}$ is called a joint probability density function or joint
pdf of the continuous bivariate random vector $(X,Y)$ if, for every $A
\in \mathcal{R}^2$:

  \begin{align*} P((X,Y) \in A) = \int_A \int f(x,y) dx dy
  \end{align*}
\end{defn}

\section{Conditional Distributions and Independence}

Often times, when we observe the value of two or more random
variables, the values observed are related.  For example, a person's
height and weight are obviously dependent.  Conditional probabilities
regarding one variable given knowledge of another variable can be
computed using the join distribution of $(X,Y)$.

\begin{defn} Let $(X,Y)$ be a discrete bivariate random vector with
joint pmf $f(x,y)$ and marginal pmfs $f_X(x)$ and $f_Y(y)$.  For any
$x$ such that $P(X = x) = f_X(x) > 0$, the conditional pmf of $Y$
given that $X = x$ is the function of $y$ denoted by $f(y|x)$ and
defined by:

  \begin{align*} f(y|x) &= P(Y = y \, | \, X = x) =
                          \frac{f(x,y)}{f_X(x)}
  \end{align*}

For any $y$ such that $P(Y = y) = f_Y(y) > 0$, the conditional pmf of
$X$ given that $Y = y$ is the function of $x$ denoted by $f(x|y)$ and
defined by:

\begin{align*} f(x|y) &= P(X = x \, | \, Y = y) =
\frac{f(x,y)}{f_Y(y)}
\end{align*}

\end{defn}

\begin{defn} Let $(X,Y)$ be a continuous bivariate random vector with
joint pdf $f(x,y)$ and marginal pdfs $f_X(x)$ and $f_Y(y)$.  For any
$x$ such that $f_X(x) > 0$, the conditional pdf of $Y$ given that $X =
x$ is the function of $y$ denoted by $f(y|x)$ and defined by:

\begin{align*} f(y|x) &= \frac{f(x,y)}{f_X(x)}
\end{align*}

\end{defn}

The conditional pmf's can also be used to calculate expected values.
IF $g(Y)$ is a function of $Y$, then the conditional expected value of
$g(Y)$ given that $X = x$ is denoted $\mathrm{E}\left[g(Y) \, | \, x
\right]$ and is given by:

\begin{align*} \mathrm{E}\left[g(Y) \, | \, x \right] &= \sum_y g(y)
f(y|x) \\ \mathrm{E}\left[g(Y) \, | \, x \right] &- \int_y g(y) f(y|x)
dy
\end{align*}

The variance of the probability distribution described by $f(y|x)$ is
called the conditional variance of $Y$ given $X = x$.  Using the
notation $Var(Y \, | \, x)$ for this, we have, using the oridinary
definition of variance,

\begin{align*} \mathrm{Var}(Y \, | \, x) &= \mathrm{E}(Y^2 \, | \, x)
- (\mathrm{E}(Y \, | \, x))^2
\end{align*}

\begin{defn} Let $(X, Y)$ be a bivariate random vector with joint pdf
or pmf $f(x,y)$ and marginal pdfs or pmfs $f_X(x)$ and $f_Y(y)$.  Then
$X$ and $Y$ are called independent random variables if, for every $x
\in \mathcal{R}$ and $y \in \mathcal{R}$,

  \begin{align*} f(x,y) &= f_X(x) f_Y(y)
  \end{align*}

  If $X$ and $Y$ are independent, the conditional pdf of $Y$ given $X
= x$ is:

  \begin{align*} f(y|x) &= \frac{f(x,y)}{f_X(x)} \\ &= \frac{f_X(x)
f_Y(y)}{f_X(x)} \\ &= f_Y(y)
  \end{align*}

  regardless of the value of $x$.
\end{defn}

\begin{thm} Let $(X,Y)$ be a bivariate random vector with joint pdf or
pmf $f(x,y)$.  Then $X$ and $Y$ are independent random variables if
and only if there exists $g(x)$ and $h(y)$ such that, for every $x \in
\mathcal{R}$ and $y \in \mathcal{R}$:
\end{thm}

\chapter{Properties of a Random Sample}

\section{Basic Concepts of Random Samples}

\begin{defn} The random variables $X_1, \dots, X_n$ are called a
random sample of size $n$ from population $f(x)$ if $X_1, \dots, X_n$
are mutually independent random variables and the marginal pdf or pmf
of each $X_i$ is the same function $f(x)$.  They are also known as
being IID.
\end{defn}

Each observation $X_i$ is on the same variable and each $X_i$ has a
marginal probability distribution given by $f(x)$.  The value of one
observation has no effect on or relationship with any of the other
observations; that is, $X_1, \dots, X_n$ are mututally independent.
The joint pdf or pmf of $X_1, \dots, X_n$ is given by:

\begin{align*} f(x_1, \dots, x_n) = f(x_1)f(x_2) \dots f(x_n) =
\prod_{i=1}^n f(x_i)
\end{align*}

If the population pdf or pmf is a member of a parameter family, the
joint pdf or pmf is then denoted:

\begin{align*} f(x_1, \dots, x_n \, | \, \theta) = \prod_{i=1}^n f(x_i
\, | \, \theta)
\end{align*}

where the same parameter value $\theta$ is used in each of the terms
in the product.  If we assume that the population follows a specified
parametetric family of distribution but the true parameter value is
unknown, then we note $\theta$ as being unknown.  By considering
different possible values of $\theta$, we can study how a random
sample would behave for different populations (e.g. maximum likelihood
parameter estimation).

\begin{ex} Let $X_1, \dots, X_n$ be a random sample from an
exponential ($\beta$) population.  Specifically, $X_1, \dots, X_n$
might correspond to the tiems until failure (measured in years) for
$n$ identical circuit boards that are put on test and used until they
fail.  The joint pdf of the sample is

  \begin{align*} f(x_1, \dots, x_n \, | \, \beta) &= \prod_{i=1}^n
f(x_i \, | \, \beta) = \prod_{i=1}^n \frac{1}{\beta} e^{-x_i / \beta}
= \frac{1}{\beta^n}e^{-(x_1 + \dots + x_n)/\beta}
  \end{align*}

  We can then answer questions about the sample.  For example, what is
the probability that all the boards last more than 2 years?  We can
compute:

  \begin{align*} P(X_1 > 2, \dots, X_n > 2) &\\ &= \int_2^{\infty}
\dots
  \end{align*}
\end{ex}

\chapter{Principles of Data Reduction}

We focus here on the task of making inferences about an unknown
parameter $\theta$.  The statistician uses the information in a sample
$X_1, \dots, X_n$ to do so.  If the sample is large enough, then it
may be unwieldy to interpret.  Said experimenter may want to summarize
the information contained within the sample by determining a few key
features of the data.  This is done by computing \textit{statistics},
or functions of the sample data.

\end{document}
